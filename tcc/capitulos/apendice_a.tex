\chapter{APÊNDICE A -- CODIGO FONTE DO PROCEDIMENTO DE TESTE DA INSERÇÃO DE NALUs SEI}

No caso dos módulos novos, todo o seu fonte é documentado aqui.

No caso dos módulos que já existiam no software de referência e sofreram alterações, apenas as funções que sofreram alteração serão documentadas aqui.


\section{Módulo adicionado ao Codificador}

\subsection{Arquivo udata\_gen.h}

\begin{lstlisting}
/*!
 *******************************************************************************
 *  \file
 *     udata_gen.h
 *  \brief
 *     definitions for Supplemental Enhanced Information Userdata Generation
 *  \author(s)
 *      - Tiago Katcipis                             <tiagokatcipis@gmail.com>
 *
 * *****************************************************************************
 */

#ifndef UDATA_GEN_H
#define UDATA_GEN_H

#include "typedefs.h"
#include "nal.h"

/*!
 *****************************************************************************************
 * Generates a SEI NALU that with ONE unregistered userdata SEI message.
 *
 * @param data The data to be sent on the SEI unregistered userdata message.
 * @param size The size of the data.
 * @return A SEI NALU containing the SEI message.
 *
 *****************************************************************************************
 */
NALU_t *user_data_generate_unregistered_sei_nalu(char * data, unsigned int size);

/*!
 *****************************************************************************************
 * Generates a SEI NALU that with ONE unregistered userdata SEI message.
 *
 * @param msg The message to be sent on the SEI unregistered userdata message, 
 *            must be a 0 terminated string.
 * @return A SEI NALU containing the SEI message.
 *
 ******************************************************************************************
 */
NALU_t *user_data_generate_unregistered_sei_nalu_from_msg(char * msg);

/*!
 *******************************************************************************************
 * Generates a 0 terminated string to be sent on a SEI NALU. The size of the message is
 * random, but it will not exceed MAXNALUSIZE.
 *
 * @return a 0 terminated string.
 *
 ********************************************************************************************
 */
char* user_data_generate_create_random_message();

/*!
 *******************************************************************************************
 * Frees the resources used by a msg.
 *
 * @param msg A previously created message.
 *
 ********************************************************************************************
 */
void user_data_generate_destroy_random_message(char * msg);

#endif
\end{lstlisting}

\subsection{Arquivo udata\_gen.c}
\begin{lstlisting}

#include <string.h>
#include <stdio.h>
#include <stdlib.h>
#include "udata_gen.h"
#include "vlc.h"
#include "sei.h"
#include "nalu.h"


static const char* random_message_start_template = "\nRandom message[%d] start\n";
static const char* random_message_end_template   = "\nRandom message[%d] end!\n";
static int msg_counter                           = 1;

/* Lets guarantee that max_msg_size + headers + 
   emulation prevention bytes dont exceed the MAXNALUSIZE */
static const int max_msg_size                    = MAXNALUSIZE - 1024; 

static int growth_rate                           = 1024;
static int actual_size                           = 1024;

#define MAX_TEMPLATE_MSG_SIZE 512
static char random_message_start_buffer[MAX_TEMPLATE_MSG_SIZE];
static char random_message_end_buffer[MAX_TEMPLATE_MSG_SIZE]; 

/*!
 *************************************************************************************
 * \brief
 *    int GenerateUserDataSEImessage_rbsp(int, byte*, char*, unsigned int)
 *
 * \return
 *    size of the RBSP in bytes, negative in case of an error
 *
 *************************************************************************************
 */
static int GenerateUserDataSEImessage_rbsp (int id, byte *rbsp, char* sei_message, unsigned int sei_message_size)
{
  Bitstream *bitstream;
  unsigned int message_size = sei_message_size;

  int LenInBytes;
  assert (rbsp != NULL);

  if ((bitstream=calloc(1, sizeof(Bitstream)))==NULL)
    no_mem_exit("SeqParameterSet:bitstream");

  // .. and use the rbsp provided (or allocated above) for the data
  bitstream->streamBuffer = rbsp;
  bitstream->bits_to_go = 8;

  {
    char uuid_message[9] = "Random"; // This is supposed to be Random
    unsigned int i;

    TIME_T start_time;
    gettime(&start_time);    // start time

    u_v (8,"SEI: last_payload_type_byte", 5, bitstream);
    message_size += 17;
    while (message_size > 254)
    {
      u_v (8,"SEI: ff_byte",255, bitstream);
      message_size -= 255;
    }
    u_v (8,"SEI: last_payload_size_byte",message_size, bitstream);

    // Lets randomize uuid based on time
    u_v (32,"SEI: uuid_iso_iec_11578",(int) start_time.tv_sec, bitstream);
    u_v (32,"SEI: uuid_iso_iec_11578",(int) start_time.tv_usec, bitstream);

    u_v (32,"SEI: uuid_iso_iec_11578",(int) (uuid_message[0] << 24) + (uuid_message[1] << 16)  + (uuid_message[2] << 8) + (uuid_message[3] << 0), bitstream);
    u_v (32,"SEI: uuid_iso_iec_11578",(int) (uuid_message[4] << 24) + (uuid_message[5] << 16)  + (uuid_message[6] << 8) + (uuid_message[7] << 0), bitstream);

    for (i = 0; i < sei_message_size; i++) {
        u_v (8,"SEI: user_data_payload_byte",sei_message[i], bitstream);
    }

    /* FIXME we MUST have this zero or the original coded was suposed to zero terminate the msg ? */
    u_v (8,"SEI: user_data_payload_byte", 0, bitstream);
  }

  SODBtoRBSP(bitstream);     // copies the last couple of bits into the byte buffer
  LenInBytes=bitstream->byte_pos;

  free(bitstream);
  return LenInBytes;
}


/*!
 *************************************************************************************
 * \brief
 *    Function body for Unregistered userdata SEI message NALU generation
 *
 * \return
 *    A NALU containing the SEI message.
 *
 *************************************************************************************
 */
NALU_t * user_data_generate_unregistered_sei_nalu(char * data, unsigned int size)
{
  NALU_t *n = AllocNALU(MAXNALUSIZE);
  int RBSPlen = 0;
  byte rbsp[MAXRBSPSIZE];

  RBSPlen = GenerateUserDataSEImessage_rbsp (NORMAL_SEI, rbsp, data, size);
  RBSPtoNALU (rbsp, n, RBSPlen, NALU_TYPE_SEI, NALU_PRIORITY_DISPOSABLE, 1);

  n->startcodeprefix_len = 4;
  return n;
}


/*!
 *************************************************************************************
 * \brief
 *    Function body for Unregistered userdata SEI message NALU generation
 *
 * \return
 *    A NALU containing the SEI message.
 *
 *************************************************************************************
 */
NALU_t *user_data_generate_unregistered_sei_nalu_from_msg(char * msg)
{
    return user_data_generate_unregistered_sei_nalu(msg, strlen(msg));
}


/*!
 *************************************************************************************
 * \brief
 *    Function body for random string message generation.
 *
 * \return
 *    A 0 terminted string.
 *
 *************************************************************************************
 */
char* user_data_generate_create_random_message()
{
    char * msg    = 0;
    int msg_size  = 0;
    int body_size = 0;
    int start_len = 0;
    int end_len   = 0;
    int i         = 0;

    /* This will help debug of the SEI messages at the decoder */
    snprintf (random_message_start_buffer, MAX_TEMPLATE_MSG_SIZE, random_message_start_template, msg_counter);
    snprintf (random_message_end_buffer, MAX_TEMPLATE_MSG_SIZE, random_message_end_template, msg_counter);
    msg_counter++;

    start_len = strlen(random_message_start_buffer);
    end_len  = strlen(random_message_end_buffer);

    body_size = actual_size;
    actual_size += growth_rate;

    if (actual_size > max_msg_size) {
        actual_size = growth_rate;
    }
    
    msg_size = body_size + start_len +  end_len + 1;
    msg      = malloc(msg_size);

    /* writing the start */
    memcpy(msg, random_message_start_buffer, start_len);
    
    /* writing the body */
    for (i = start_len; i < msg_size - end_len; i++) {
        msg[i] = 'h'; /* whatever */
    }

    /* writing the end */
    memcpy(msg + start_len + body_size, random_message_end_buffer, end_len);

    /* 0 terminate it */
    msg[msg_size - 1] = '\0';
    return msg;
}

/*!
 *************************************************************************************
 * \brief
 *    Function body for random message destruction.
 *************************************************************************************
 */
void user_data_generate_destroy_random_message(char * msg)
{
    free(msg);
}
\end{lstlisting}

\section{Alterações no Codificador}

\subsection{Arquivo filehandle.c}
\begin{lstlisting}

/*!
 ************************************************************************
 * \brief
 *    This function opens the output files and generates the
 *    appropriate sequence header
 ************************************************************************
 */
int start_sequence(VideoParameters *p_Vid, InputParameters *p_Inp)
{
  int i,len=0, total_pps = (p_Inp->GenerateMultiplePPS) ? 3 : 1;
  NALU_t *nalu;

  switch(p_Inp->of_mode)
  {
    case PAR_OF_ANNEXB:
      OpenAnnexbFile (p_Vid, p_Inp->outfile);
      p_Vid->WriteNALU = WriteAnnexbNALU;
      break;
    case PAR_OF_RTP:
      OpenRTPFile (p_Vid, p_Inp->outfile);
      p_Vid->WriteNALU = WriteRTPNALU;
      break;
    default:
      snprintf(errortext, ET_SIZE, "Output File Mode %d not supported", p_Inp->of_mode);
      error(errortext,1);
  }

  // Access Unit Delimiter NALU
  if ( p_Inp->SendAUD )
  {
    len += Write_AUD_NALU(p_Vid);
  }

  //! As a sequence header, here we write both sequence and picture
  //! parameter sets.  As soon as IDR is implemented, this should go to the
  //! IDR part, as both parsets have to be transmitted as part of an IDR.
  //! An alternative may be to consider this function the IDR start function.

  nalu = NULL;
  nalu = GenerateSeq_parameter_set_NALU (p_Vid);
  len += p_Vid->WriteNALU (p_Vid, nalu);
  FreeNALU (nalu);

#if (MVC_EXTENSION_ENABLE)
  if(p_Inp->num_of_views==2)
  {
    int bits;
    nalu = NULL;
    nalu = GenerateSubsetSeq_parameter_set_NALU (p_Vid);
    bits = p_Vid->WriteNALU (p_Vid, nalu);
    len += bits;
    p_Vid->p_Stats->bit_ctr_parametersets_n_v[1] = bits;
    FreeNALU (nalu);
  }
  else
  {
    p_Vid->p_Stats->bit_ctr_parametersets_n_v[1] = 0;
  }
#endif

  //! Lets write now the Picture Parameter sets. Output will be equal to the total number of bits spend here.
  for (i=0;i<total_pps;i++)
  {
     len = write_PPS(p_Vid, len, i);
  }

  if (p_Inp->GenerateSEIMessage)
  {
    nalu = NULL;
    nalu = GenerateSEImessage_NALU(p_Inp);
    len += p_Vid->WriteNALU (p_Vid, nalu);
    FreeNALU (nalu);
  }

  /* Lets send 1000 SEI NALUs containing a SEI Userdata Unregistered message */
  for (i = 0; i < 1000; i++) {
      nalu = NULL;
      char * msg = user_data_generate_create_random_message();
      nalu = user_data_generate_unregistered_sei_nalu_from_msg(msg);
      len += p_Vid->WriteNALU (p_Vid, nalu);
      FreeNALU (nalu);
      user_data_generate_destroy_random_message(msg);
  }


  p_Vid->p_Stats->bit_ctr_parametersets_n = len;
#if (MVC_EXTENSION_ENABLE)
  if(p_Inp->num_of_views==2)
  {
    p_Vid->p_Stats->bit_ctr_parametersets_n_v[0] = len - p_Vid->p_Stats->bit_ctr_parametersets_n_v[1];
  }
#endif
  return 0;
}
\end{lstlisting}

\subsection{Arquivo lencod.c}
\begin{lstlisting}
/*!
 ***********************************************************************
 * \brief
 *    Encode a sequence
 ***********************************************************************
 */
static void encode_sequence(VideoParameters *p_Vid, InputParameters *p_Inp)
{
  int curr_frame_to_code;
  int frames_to_code;
  int frame_num_bak = 0, frame_coded;
  int frm_struct_buffer;
  SeqStructure *p_seq_struct = p_Vid->p_pred;
  FrameUnitStruct *p_frm;

#if (MVC_EXTENSION_ENABLE)
  int tmp_rate_control_enable = p_Inp->RCEnable;

  if ( p_Inp->num_of_views == 2 )
  {
    frames_to_code = p_Inp->no_frames << 1;
    p_frm = p_seq_struct->p_frm_mvc;
    frm_struct_buffer = p_seq_struct->num_frames_mvc;
  }
  else
#endif
  {
    frames_to_code = p_Inp->no_frames;
    p_frm = p_seq_struct->p_frm;
    frm_struct_buffer = p_Vid->frm_struct_buffer;
  }
  
  for (curr_frame_to_code = 0; curr_frame_to_code < frames_to_code; curr_frame_to_code++)
  {
#if (MVC_EXTENSION_ENABLE)
    if ( p_Inp->num_of_views == 2 )
    {
      if ( (curr_frame_to_code & 1) == 0 ) // call only for view_id 0
      {
        // determine whether to populate additional frames in the prediction structure
        if ( (curr_frame_to_code >> 1) >= p_Vid->p_pred->pop_start_frame )
        {
          int start = p_seq_struct->pop_start_frame, end;

          populate_frm_struct( p_Vid, p_Inp, p_seq_struct, p_Inp->FrmStructBufferLength, frames_to_code >> 1 );
          end = p_seq_struct->pop_start_frame;
          populate_frm_struct_mvc( p_Vid, p_Inp, p_seq_struct, start, end );
        }
      }
    }
    else
#endif
    {
      // determine whether to populate additional frames in the prediction structure
      if ( curr_frame_to_code >= p_Vid->p_pred->pop_start_frame )
      {
        populate_frm_struct( p_Vid, p_Inp, p_seq_struct, p_Inp->FrmStructBufferLength, frames_to_code );
      }
    }

    p_Vid->curr_frm_idx = curr_frame_to_code;
    p_Vid->p_curr_frm_struct = p_frm + ( p_Vid->curr_frm_idx % frm_struct_buffer ); // pointer to current frame structure
    p_Vid->number = curr_frame_to_code;

#if (MVC_EXTENSION_ENABLE)
    if(p_Inp->num_of_views==2)
    {
      p_Vid->view_id = p_Vid->p_curr_frm_struct->view_id;
      if ( p_Vid->view_id == 1 )
      {
        p_Vid->curr_frm_idx = p_Vid->number = (curr_frame_to_code - 1) >> 1;
        p_Vid->p_curr_frm_struct->qp = p_Vid->qp = iClip3( -p_Vid->bitdepth_luma_qp_scale, MAX_QP, p_Vid->AverageFrameQP + p_Inp->View1QPOffset );
      }
      else
      {
        p_Vid->curr_frm_idx = p_Vid->number = curr_frame_to_code >> 1;
      }
      if ( p_Vid->view_id == 1 && tmp_rate_control_enable )
      {
        p_Inp->RCEnable = 0;        
      }
      else
      {
        p_Inp->RCEnable = tmp_rate_control_enable;
      }
    }
#endif

    if ( p_Vid->p_curr_frm_struct->frame_no >= p_Inp->no_frames )
    {
      continue;
    }

    // Update frame_num counter
    frame_num_bak = p_Vid->frame_num;
    if (p_Vid->last_ref_idc == 1)
    {      
      p_Vid->frame_num++;

#if (MVC_EXTENSION_ENABLE)
      if ( p_Inp->num_of_views == 2 )
      {
        p_Vid->frame_num %= (p_Vid->max_frame_num << 1);
      }
      else
#endif
      p_Vid->frame_num %= p_Vid->max_frame_num;
    }

    prepare_frame_params(p_Vid, p_Inp, curr_frame_to_code);

    // redundant frame initialization and allocation
    if (p_Inp->redundant_pic_flag)
    {
      init_redundant_frame(p_Vid, p_Inp);
      set_redundant_frame(p_Vid, p_Inp);
    }

    frame_coded = encode_one_frame(p_Vid, p_Inp); // encode one frame;
    if ( !frame_coded )
    {
      p_Vid->frame_num = frame_num_bak;
      continue;
    }

    p_Vid->last_ref_idc = p_Vid->nal_reference_idc ? 1 : 0;

    // if key frame is encoded, encode one redundant frame
    if (p_Inp->redundant_pic_flag && p_Vid->key_frame)
    {
      encode_one_redundant_frame(p_Vid, p_Inp);
    }

    if (p_Vid->type == I_SLICE && p_Inp->EnableOpenGOP)
      p_Vid->last_valid_reference = p_Vid->ThisPOC;

    if (p_Inp->ReportFrameStats)
      report_frame_statistic(p_Vid, p_Inp);

    /* Inserting a SEI NALU (Userdata Unregistered) */
    char * msg   = user_data_generate_create_random_message();
    NALU_t *nalu = user_data_generate_unregistered_sei_nalu_from_msg(msg);
    p_Vid->WriteNALU (p_Vid, nalu);
    FreeNALU (nalu);
    user_data_generate_destroy_random_message(msg);
  }

#if EOS_OUTPUT
  end_of_stream(p_Vid);
#endif

#if (MVC_EXTENSION_ENABLE)
  if(p_Inp->num_of_views == 2)
  {
    p_Inp->RCEnable = tmp_rate_control_enable;
  }
#endif
}
\end{lstlisting}

\section{Módulo adicionado ao Decodificador}

\subsection{Arquivo udata\_parser.h}
\begin{lstlisting}
/*!
 ************************************************************************
 *  \file
 *     udata_parser.h
 *  \brief
 *     definitions for Supplemental Enhanced Information Userdata Parsing
 *  \author(s)
 *      - Tiago Katcipis                             <tiagokatcipis@gmail.com>
 *
 * ************************************************************************
 */

#ifndef UDATA_PARSER_H
#define UDATA_PARSER_H

#include "typedefs.h"

/*!
 *******************************************************************************
 * Parses a SEI User Data Unregistered message, dumping all its data at stdout.
 *
 * @param payload The payload of the SEI message.
 * @param size The size of the payload.
 *
 *******************************************************************************
 */
void user_data_parser_unregistered_sei(byte* payload, int size);

#endif
\end{lstlisting}

\subsection{Arquivo udata\_parser.c}
\begin{lstlisting}
#include "udata_parser.h"
#include <stdio.h>

#define UUID_ISO_IEC_OFFSET 16 


void user_data_parser_unregistered_sei( byte* payload, int size)
{
  int offset = 0;
  byte payload_byte;

  printf("\nUser data unregistered SEI message, size[%d]\n", size);
  printf("uuid_iso_11578 = 0x");
  
  assert (size>=UUID_ISO_IEC_OFFSET);

  for (offset = 0; offset < UUID_ISO_IEC_OFFSET; offset++)
  {
    printf("%02x",payload[offset]);
  }

  printf("\nUser data unregistered SEI message start\n");

  while (offset < size)
  {
    payload_byte = payload[offset];
    offset ++;
    printf("%c", payload_byte);
  }

  printf("\nUser data unregistered SEI message end \n");
}
\end{lstlisting}

\section{Alterações no Decodificador}

\subsection{Arquivo sei.c}
\begin{lstlisting}
void InterpretSEIMessage(byte* msg, int size, VideoParameters *p_Vid, Slice *pSlice)
{
  int payload_type = 0;
  int payload_size = 0;
  int offset = 1;
  byte tmp_byte;
  
  do
  {
    // sei_message();
    payload_type = 0;
    tmp_byte = msg[offset++];
    while (tmp_byte == 0xFF)
    {
      payload_type += 255;
      tmp_byte = msg[offset++];
    }
    payload_type += tmp_byte;   // this is the last byte

    payload_size = 0;
    tmp_byte = msg[offset++];
    while (tmp_byte == 0xFF)
    {
      payload_size += 255;
      tmp_byte = msg[offset++];
    }
    payload_size += tmp_byte;   // this is the last byte

    switch ( payload_type )     // sei_payload( type, size );
    {
    case  SEI_BUFFERING_PERIOD:
      interpret_buffering_period_info( msg+offset, payload_size, p_Vid );
      break;
    case  SEI_PIC_TIMING:
      interpret_picture_timing_info( msg+offset, payload_size, p_Vid );
      break;
    case  SEI_PAN_SCAN_RECT:
      interpret_pan_scan_rect_info( msg+offset, payload_size, p_Vid );
      break;
    case  SEI_FILLER_PAYLOAD:
      interpret_filler_payload_info( msg+offset, payload_size, p_Vid );
      break;
    case  SEI_USER_DATA_REGISTERED_ITU_T_T35:
      interpret_user_data_registered_itu_t_t35_info( msg+offset, payload_size, p_Vid );
      break;
    case  SEI_USER_DATA_UNREGISTERED:
      user_data_parser_unregistered_sei(msg+offset, payload_size);
      break;
    case  SEI_RECOVERY_POINT:
      interpret_recovery_point_info( msg+offset, payload_size, p_Vid );
      break;
    case  SEI_DEC_REF_PIC_MARKING_REPETITION:
      interpret_dec_ref_pic_marking_repetition_info( msg+offset, payload_size, p_Vid, pSlice );
      break;
    case  SEI_SPARE_PIC:
      interpret_spare_pic( msg+offset, payload_size, p_Vid );
      break;
    case  SEI_SCENE_INFO:
      interpret_scene_information( msg+offset, payload_size, p_Vid );
      break;
    case  SEI_SUB_SEQ_INFO:
      interpret_subsequence_info( msg+offset, payload_size, p_Vid );
      break;
    case  SEI_SUB_SEQ_LAYER_CHARACTERISTICS:
      interpret_subsequence_layer_characteristics_info( msg+offset, payload_size, p_Vid );
      break;
    case  SEI_SUB_SEQ_CHARACTERISTICS:
      interpret_subsequence_characteristics_info( msg+offset, payload_size, p_Vid );
      break;
    case  SEI_FULL_FRAME_FREEZE:
      interpret_full_frame_freeze_info( msg+offset, payload_size, p_Vid );
      break;
    case  SEI_FULL_FRAME_FREEZE_RELEASE:
      interpret_full_frame_freeze_release_info( msg+offset, payload_size, p_Vid );
      break;
    case  SEI_FULL_FRAME_SNAPSHOT:
      interpret_full_frame_snapshot_info( msg+offset, payload_size, p_Vid );
      break;
    case  SEI_PROGRESSIVE_REFINEMENT_SEGMENT_START:
      interpret_progressive_refinement_start_info( msg+offset, payload_size, p_Vid );
      break;
    case  SEI_PROGRESSIVE_REFINEMENT_SEGMENT_END:
      interpret_progressive_refinement_end_info( msg+offset, payload_size, p_Vid );
      break;
    case  SEI_MOTION_CONSTRAINED_SLICE_GROUP_SET:
      interpret_motion_constrained_slice_group_set_info( msg+offset, payload_size, p_Vid );
    case  SEI_FILM_GRAIN_CHARACTERISTICS:
      interpret_film_grain_characteristics_info ( msg+offset, payload_size, p_Vid );
      break;
    case  SEI_DEBLOCKING_FILTER_DISPLAY_PREFERENCE:
      interpret_deblocking_filter_display_preference_info ( msg+offset, payload_size, p_Vid );
      break;
    case  SEI_STEREO_VIDEO_INFO:
      interpret_stereo_video_info_info ( msg+offset, payload_size, p_Vid );
      break;
    case  SEI_TONE_MAPPING:
      interpret_tone_mapping( msg+offset, payload_size, p_Vid );
      break;
    case  SEI_POST_FILTER_HINTS:
      interpret_post_filter_hints_info ( msg+offset, payload_size, p_Vid );
      break;
    case  SEI_FRAME_PACKING_ARRANGEMENT:
      interpret_frame_packing_arrangement_info( msg+offset, payload_size, p_Vid );
      break;
    default:
      interpret_reserved_info( msg+offset, payload_size, p_Vid );
      break;    
    }
    offset += payload_size;

  } while( msg[offset] != 0x80 );    // more_rbsp_data()  msg[offset] != 0x80
  // ignore the trailing bits rbsp_trailing_bits();
  assert(msg[offset] == 0x80);      // this is the trailing bits
  assert( offset+1 == size );
}

\end{lstlisting}

