\chapter*{Agradecimentos}

Primeiramente gostaria de agradecer à minha amada esposa e melhor amiga Stephanie, que além de me mostrar o real significado do amor e me apresentar a felicidade de uma maneira que eu nao conhecia, cooperou no desenvolvimento do trabalho desde o seu início, suportando amorosamente a minha ausência, se sobrecarregando com trabalho (inclusive o de revisar esse texto) permitindo que eu me dedicasse integralmente ao desenvolvimento do mesmo.

Aos meus pais agradeço por todo o esforço e sacrifício que fizeram para que eu pudesse chegar aqui, sem eles eu certamente não teria oportunidade de realizar este trabalho nem de viver todas as experiências maravilhosas que tive até hoje.

Ao professor Guto agradeço pela ajuda na escolha de um tema que além de interessante também é relacionado ao meu trabalho, e pela orientação em momentos críticos, onde o caminho a se seguir não era tão claro. 

Pela oportunidade de desenvolver este trabalho em parceria com a Dígitro agradeço ao Rafael Pina que acreditou na proposta do trabalho e ao Paulo Pizarro, que além de acreditar no trabalho, co-orientou ele, fornecendo ajuda valiosa em decisões críticas ao longo do projeto e na formulação do texto final.

Agradeço ao Mateus Ludwich pela ajuda fundamental no desenvolvimento do trabalho, ajudando a escolher o melhor rumo a seguir em momentos críticos, esclarecendo dúvidas a respeito do software de referência do H.264, revisando o texto, revisando apresentações, apontando melhoramentos que podiam ser feitos, certamente o resultado não teria sido o mesmo sem a sua ajuda.

Ao Alexis Tourapis devo um agradecimento pela ajuda no entendimento de estruturas de dados e funções importantes do codificador do software de referência do H.264.

Acima de tudo agradeço a Jeová Deus, o dador de "toda boa dádiva e todo presente perfeito". (Tiago 1:17)


